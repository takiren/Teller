\documentclass[12pt,a4paper,uplatex,dvipdfmx]{jsarticle}
\usepackage{titlesec}
\usepackage{listings,jvlisting}

\usepackage[top=30truemm,bottom=30truemm,left=25truemm,right=25truemm,truedimen]{geometry}

%ここからソースコードの表示に関する設定
\lstset{
  basicstyle={\ttfamily},
  identifierstyle={\small},
  commentstyle={\smallitshape},
  keywordstyle={\small\bfseries},
  ndkeywordstyle={\small},
  stringstyle={\small\ttfamily},
  frame={tb},
  breaklines=true,
  columns=[l]{fullflexible},
  numbers=left,
  xrightmargin=0zw,
  xleftmargin=3zw,
  numberstyle={\scriptsize},
  stepnumber=1,
  numbersep=1zw,
  lineskip=-0.5ex
}
%ここまでソースコードの表示に関する設定
\titleformat*{\section}{\Huge}
\begin{document}
\title{Teller Engineドキュメント\\ \bfseries{執筆中}}
\author{\texttt{たき}}
\date{最終更新日\today}

\maketitle
\section*{はじめに}
みなさんこんにちは、たきです。「たき」や「たきれん」など、複数名前がありますがたきです。このドキュメントはTellerEngineを改造したいだとか、ちょっと参考にしたいだとかを考えている人に向けた文書です。
本書ではどのように設計したのか使用しているライブラリの話だとかをまとめたいと思います。
説明不足の点もあると思いますが気付き次第改善する所存でありますので、どうかよろしくお願い致します。
\newpage
\tableofcontents
\newpage
\section{TellerEngineことはじめ}
\subsection{外部ライブラリ}
では記念すべき本文を書いていきましょう。まずはじめに使用している外部ライブラリのお話をします。以下のリストが使用している外部ライブラリです。

\begin{quote}
  \begin{itemize}
    \item Cinder 描画ライブラリ。他にもいろいろな機能があるが描画と画像のロードを担当している。imgui-node-editorを組み込んだカスタムビルド。
    \item imgui-node-editor imguiでノードエディタを作れる拡張。Cinderライブラリのビルド時に一緒に入れた。
    \item nlohmann\_json json用パーサー。速くて有名な実績のあるパーサー。
  \end{itemize}
\end{quote}
他にもライブラリが見えていると思いますが使いません。
\subsection{必要なもの}
基本的にC++で書いていて今のところwindowsでしか動きません、おそらく。CMakeListsも書きかけのためまだLinux、Macの対応は先でしょう。
というわけでまずwindowsマシンです。それ以外は知りません。
筆者の環境をとりあえず載せます。
\begin{quote}
  \begin{itemize}
    \item Windows 10 21H2 build 19044.1706
    \item Visual Studio 2022 Community
    \item 標準C++17
    \item Git
  \end{itemize}
\end{quote}
以上があればビルドは通せるはずです。


\subsection{はじめの一歩}
まずCinderからビルドを通しましょう。Githubからクローンした場合既にimgui-node-editorもプロジェクトに入っているはずなので/Cinder/build/cinder.slnを開いて、DebugとReleaseのそれぞれでビルドしましょう。
ビルドが終わったら/TellerEngin/TellerEngine.slnを開いてください。ビルドが通ったら準備完了です。
\section{TellerEnginの設計}
\subsection{モジュール}
TellerEngineは以下の基本的なクラスで構成されています。
\begin{quote}
  \begin{itemize}
    \item TellerCore
    \item Editor 及びその派生クラス
    \item ModuleCore 及びその派生クラス
  \end{itemize}
\end{quote}
\subsubsection{TellerCore}
TellerCoreは上記のクラス、及びそれ以外のAnimationクラスのなどを管理する中枢となるクラスです。このクラスからModuleCoreの派生クラスであったりEditorクラスを呼び出します。
\subsubsection{ModuleCore}
モジュールの基底クラス。詳細は後述。
\subsubsection{Editor}
エディターの基底クラス。

\subsubsection{SceneModule}
シーンを構成するモジュールです。具体的には場面やCGシーンなどを移すときに使うことになります。このモジュールをGameModuleにキューとして追加していきます。


\section{命名規則}
\subsection*{クラス}
\begin{quote}
  \begin{itemize}
    \item 大文字から始める。
    \item クラスを継承している場合は継承元のクラス名を入れる。
  \end{itemize}
\end{quote}
\subsection*{関数}
\begin{quote}
  \begin{itemize}
    \item 
  \end{itemize}
\end{quote}
\subsection*{変数}

\section{簡単な使い方}
\subsection{プロジェクト作成}
ではこの章から実際にビルドまでしてみましょう。
せっかくなのでプロジェクトから作ってみます。
VisualStudioでC++から空のプロジェクトを作ってください。
C++の言語標準はC++17。
ランタイムライブラリをdebug、Releaseでそれぞれマルチスレッドデバッグ、マルチスレッドに設定。
追加のインクルードディレクトリにCinder/includeとTellerEngine/srcを追加。
追加のライブラリディレクトリにcinder.libのあるパスを追加します。
その後追加の依存ファイルにcinder.libを追加します。
サブシステムをWindowsに設定。
以上で終了です。

\subsection{Cinder}
Cinderはマルチプラットフォーム対応の描画ライブラリのためプログラマーがWinMain関数を書く必要はありません。その代わりCinderのAppクラスを継承したクラスがWinMain関数の代わりとなります。

ではまずCinder::Appを継承したクラスを作ります。
main.cppを作成してください。
そして以下のコードを書きます。

\begin{lstlisting}[caption=hoge,label=fuga]
#include "cinder/app/App.h"
#include "cinder/app/RendererGl.h"
#include "cinder/gl/gl.h"
class TellerEngineMain : public App {
public:
	std::shared_ptr<TellerCore> mCore;

	void setup() override;

	void draw() override;
};

void setup()
{
}

void BasicApp::draw()
{
	gl::clear( Color::gray( 0.1f ) );
	gl::end();
}
CINDER_APP(TellerEngineMain, RendererGl)
\end{lstlisting}

ビルドが通ったら次に行きます。
\subsection{TellerCoreからSceneまで}

\Large{工事中}
\normalsize

\section{}

\newpage
\appendix
\section*{謝辞}
このエンジンを開発していく中で先駆者様のお力添えを何度もいただきました、また頂いております。貴重な情報をブログ等にまとめてくださりありがとうございます。

\subsection*{連絡先}
e-mail : tyorokyuu@gmail.com

\end{document}